\documentclass[11pt]{article}

    \usepackage[breakable]{tcolorbox}
    \usepackage{parskip} % Stop auto-indenting (to mimic markdown behaviour)
    
    \usepackage{iftex}
    \ifPDFTeX
    	\usepackage[T1]{fontenc}
    	\usepackage{mathpazo}
    \else
    	\usepackage{fontspec}
    \fi

    % Basic figure setup, for now with no caption control since it's done
    % automatically by Pandoc (which extracts ![](path) syntax from Markdown).
    \usepackage{graphicx}
    % Maintain compatibility with old templates. Remove in nbconvert 6.0
    \let\Oldincludegraphics\includegraphics
    % Ensure that by default, figures have no caption (until we provide a
    % proper Figure object with a Caption API and a way to capture that
    % in the conversion process - todo).
    \usepackage{caption}
    \DeclareCaptionFormat{nocaption}{}
    \captionsetup{format=nocaption,aboveskip=0pt,belowskip=0pt}

    \usepackage[Export]{adjustbox} % Used to constrain images to a maximum size
    \adjustboxset{max size={0.9\linewidth}{0.9\paperheight}}
    \usepackage{float}
    \floatplacement{figure}{H} % forces figures to be placed at the correct location
    \usepackage{xcolor} % Allow colors to be defined
    \usepackage{enumerate} % Needed for markdown enumerations to work
    \usepackage{geometry} % Used to adjust the document margins
    \usepackage{amsmath} % Equations
    \usepackage{amssymb} % Equations
    \usepackage{textcomp} % defines textquotesingle
    % Hack from http://tex.stackexchange.com/a/47451/13684:
    \AtBeginDocument{%
        \def\PYZsq{\textquotesingle}% Upright quotes in Pygmentized code
    }
    \usepackage{upquote} % Upright quotes for verbatim code
    \usepackage{eurosym} % defines \euro
    \usepackage[mathletters]{ucs} % Extended unicode (utf-8) support
    \usepackage{fancyvrb} % verbatim replacement that allows latex
    \usepackage{grffile} % extends the file name processing of package graphics 
                         % to support a larger range
    \makeatletter % fix for grffile with XeLaTeX
    \def\Gread@@xetex#1{%
      \IfFileExists{"\Gin@base".bb}%
      {\Gread@eps{\Gin@base.bb}}%
      {\Gread@@xetex@aux#1}%
    }
    \makeatother

    % The hyperref package gives us a pdf with properly built
    % internal navigation ('pdf bookmarks' for the table of contents,
    % internal cross-reference links, web links for URLs, etc.)
    \usepackage{hyperref}
    % The default LaTeX title has an obnoxious amount of whitespace. By default,
    % titling removes some of it. It also provides customization options.
    \usepackage{titling}
    \usepackage{longtable} % longtable support required by pandoc >1.10
    \usepackage{booktabs}  % table support for pandoc > 1.12.2
    \usepackage[inline]{enumitem} % IRkernel/repr support (it uses the enumerate* environment)
    \usepackage[normalem]{ulem} % ulem is needed to support strikethroughs (\sout)
                                % normalem makes italics be italics, not underlines
    \usepackage{mathrsfs}
    

    
    % Colors for the hyperref package
    \definecolor{urlcolor}{rgb}{0,.145,.698}
    \definecolor{linkcolor}{rgb}{.71,0.21,0.01}
    \definecolor{citecolor}{rgb}{.12,.54,.11}

    % ANSI colors
    \definecolor{ansi-black}{HTML}{3E424D}
    \definecolor{ansi-black-intense}{HTML}{282C36}
    \definecolor{ansi-red}{HTML}{E75C58}
    \definecolor{ansi-red-intense}{HTML}{B22B31}
    \definecolor{ansi-green}{HTML}{00A250}
    \definecolor{ansi-green-intense}{HTML}{007427}
    \definecolor{ansi-yellow}{HTML}{DDB62B}
    \definecolor{ansi-yellow-intense}{HTML}{B27D12}
    \definecolor{ansi-blue}{HTML}{208FFB}
    \definecolor{ansi-blue-intense}{HTML}{0065CA}
    \definecolor{ansi-magenta}{HTML}{D160C4}
    \definecolor{ansi-magenta-intense}{HTML}{A03196}
    \definecolor{ansi-cyan}{HTML}{60C6C8}
    \definecolor{ansi-cyan-intense}{HTML}{258F8F}
    \definecolor{ansi-white}{HTML}{C5C1B4}
    \definecolor{ansi-white-intense}{HTML}{A1A6B2}
    \definecolor{ansi-default-inverse-fg}{HTML}{FFFFFF}
    \definecolor{ansi-default-inverse-bg}{HTML}{000000}

    % commands and environments needed by pandoc snippets
    % extracted from the output of `pandoc -s`
    \providecommand{\tightlist}{%
      \setlength{\itemsep}{0pt}\setlength{\parskip}{0pt}}
    \DefineVerbatimEnvironment{Highlighting}{Verbatim}{commandchars=\\\{\}}
    % Add ',fontsize=\small' for more characters per line
    \newenvironment{Shaded}{}{}
    \newcommand{\KeywordTok}[1]{\textcolor[rgb]{0.00,0.44,0.13}{\textbf{{#1}}}}
    \newcommand{\DataTypeTok}[1]{\textcolor[rgb]{0.56,0.13,0.00}{{#1}}}
    \newcommand{\DecValTok}[1]{\textcolor[rgb]{0.25,0.63,0.44}{{#1}}}
    \newcommand{\BaseNTok}[1]{\textcolor[rgb]{0.25,0.63,0.44}{{#1}}}
    \newcommand{\FloatTok}[1]{\textcolor[rgb]{0.25,0.63,0.44}{{#1}}}
    \newcommand{\CharTok}[1]{\textcolor[rgb]{0.25,0.44,0.63}{{#1}}}
    \newcommand{\StringTok}[1]{\textcolor[rgb]{0.25,0.44,0.63}{{#1}}}
    \newcommand{\CommentTok}[1]{\textcolor[rgb]{0.38,0.63,0.69}{\textit{{#1}}}}
    \newcommand{\OtherTok}[1]{\textcolor[rgb]{0.00,0.44,0.13}{{#1}}}
    \newcommand{\AlertTok}[1]{\textcolor[rgb]{1.00,0.00,0.00}{\textbf{{#1}}}}
    \newcommand{\FunctionTok}[1]{\textcolor[rgb]{0.02,0.16,0.49}{{#1}}}
    \newcommand{\RegionMarkerTok}[1]{{#1}}
    \newcommand{\ErrorTok}[1]{\textcolor[rgb]{1.00,0.00,0.00}{\textbf{{#1}}}}
    \newcommand{\NormalTok}[1]{{#1}}
    
    % Additional commands for more recent versions of Pandoc
    \newcommand{\ConstantTok}[1]{\textcolor[rgb]{0.53,0.00,0.00}{{#1}}}
    \newcommand{\SpecialCharTok}[1]{\textcolor[rgb]{0.25,0.44,0.63}{{#1}}}
    \newcommand{\VerbatimStringTok}[1]{\textcolor[rgb]{0.25,0.44,0.63}{{#1}}}
    \newcommand{\SpecialStringTok}[1]{\textcolor[rgb]{0.73,0.40,0.53}{{#1}}}
    \newcommand{\ImportTok}[1]{{#1}}
    \newcommand{\DocumentationTok}[1]{\textcolor[rgb]{0.73,0.13,0.13}{\textit{{#1}}}}
    \newcommand{\AnnotationTok}[1]{\textcolor[rgb]{0.38,0.63,0.69}{\textbf{\textit{{#1}}}}}
    \newcommand{\CommentVarTok}[1]{\textcolor[rgb]{0.38,0.63,0.69}{\textbf{\textit{{#1}}}}}
    \newcommand{\VariableTok}[1]{\textcolor[rgb]{0.10,0.09,0.49}{{#1}}}
    \newcommand{\ControlFlowTok}[1]{\textcolor[rgb]{0.00,0.44,0.13}{\textbf{{#1}}}}
    \newcommand{\OperatorTok}[1]{\textcolor[rgb]{0.40,0.40,0.40}{{#1}}}
    \newcommand{\BuiltInTok}[1]{{#1}}
    \newcommand{\ExtensionTok}[1]{{#1}}
    \newcommand{\PreprocessorTok}[1]{\textcolor[rgb]{0.74,0.48,0.00}{{#1}}}
    \newcommand{\AttributeTok}[1]{\textcolor[rgb]{0.49,0.56,0.16}{{#1}}}
    \newcommand{\InformationTok}[1]{\textcolor[rgb]{0.38,0.63,0.69}{\textbf{\textit{{#1}}}}}
    \newcommand{\WarningTok}[1]{\textcolor[rgb]{0.38,0.63,0.69}{\textbf{\textit{{#1}}}}}
    
    
    % Define a nice break command that doesn't care if a line doesn't already
    % exist.
    \def\br{\hspace*{\fill} \\* }
    % Math Jax compatibility definitions
    \def\gt{>}
    \def\lt{<}
    \let\Oldtex\TeX
    \let\Oldlatex\LaTeX
    \renewcommand{\TeX}{\textrm{\Oldtex}}
    \renewcommand{\LaTeX}{\textrm{\Oldlatex}}
    % Document parameters
    % Document title
    \title{NumpyDemoWeek2}
    
    
    
    
    
% Pygments definitions
\makeatletter
\def\PY@reset{\let\PY@it=\relax \let\PY@bf=\relax%
    \let\PY@ul=\relax \let\PY@tc=\relax%
    \let\PY@bc=\relax \let\PY@ff=\relax}
\def\PY@tok#1{\csname PY@tok@#1\endcsname}
\def\PY@toks#1+{\ifx\relax#1\empty\else%
    \PY@tok{#1}\expandafter\PY@toks\fi}
\def\PY@do#1{\PY@bc{\PY@tc{\PY@ul{%
    \PY@it{\PY@bf{\PY@ff{#1}}}}}}}
\def\PY#1#2{\PY@reset\PY@toks#1+\relax+\PY@do{#2}}

\expandafter\def\csname PY@tok@w\endcsname{\def\PY@tc##1{\textcolor[rgb]{0.73,0.73,0.73}{##1}}}
\expandafter\def\csname PY@tok@c\endcsname{\let\PY@it=\textit\def\PY@tc##1{\textcolor[rgb]{0.25,0.50,0.50}{##1}}}
\expandafter\def\csname PY@tok@cp\endcsname{\def\PY@tc##1{\textcolor[rgb]{0.74,0.48,0.00}{##1}}}
\expandafter\def\csname PY@tok@k\endcsname{\let\PY@bf=\textbf\def\PY@tc##1{\textcolor[rgb]{0.00,0.50,0.00}{##1}}}
\expandafter\def\csname PY@tok@kp\endcsname{\def\PY@tc##1{\textcolor[rgb]{0.00,0.50,0.00}{##1}}}
\expandafter\def\csname PY@tok@kt\endcsname{\def\PY@tc##1{\textcolor[rgb]{0.69,0.00,0.25}{##1}}}
\expandafter\def\csname PY@tok@o\endcsname{\def\PY@tc##1{\textcolor[rgb]{0.40,0.40,0.40}{##1}}}
\expandafter\def\csname PY@tok@ow\endcsname{\let\PY@bf=\textbf\def\PY@tc##1{\textcolor[rgb]{0.67,0.13,1.00}{##1}}}
\expandafter\def\csname PY@tok@nb\endcsname{\def\PY@tc##1{\textcolor[rgb]{0.00,0.50,0.00}{##1}}}
\expandafter\def\csname PY@tok@nf\endcsname{\def\PY@tc##1{\textcolor[rgb]{0.00,0.00,1.00}{##1}}}
\expandafter\def\csname PY@tok@nc\endcsname{\let\PY@bf=\textbf\def\PY@tc##1{\textcolor[rgb]{0.00,0.00,1.00}{##1}}}
\expandafter\def\csname PY@tok@nn\endcsname{\let\PY@bf=\textbf\def\PY@tc##1{\textcolor[rgb]{0.00,0.00,1.00}{##1}}}
\expandafter\def\csname PY@tok@ne\endcsname{\let\PY@bf=\textbf\def\PY@tc##1{\textcolor[rgb]{0.82,0.25,0.23}{##1}}}
\expandafter\def\csname PY@tok@nv\endcsname{\def\PY@tc##1{\textcolor[rgb]{0.10,0.09,0.49}{##1}}}
\expandafter\def\csname PY@tok@no\endcsname{\def\PY@tc##1{\textcolor[rgb]{0.53,0.00,0.00}{##1}}}
\expandafter\def\csname PY@tok@nl\endcsname{\def\PY@tc##1{\textcolor[rgb]{0.63,0.63,0.00}{##1}}}
\expandafter\def\csname PY@tok@ni\endcsname{\let\PY@bf=\textbf\def\PY@tc##1{\textcolor[rgb]{0.60,0.60,0.60}{##1}}}
\expandafter\def\csname PY@tok@na\endcsname{\def\PY@tc##1{\textcolor[rgb]{0.49,0.56,0.16}{##1}}}
\expandafter\def\csname PY@tok@nt\endcsname{\let\PY@bf=\textbf\def\PY@tc##1{\textcolor[rgb]{0.00,0.50,0.00}{##1}}}
\expandafter\def\csname PY@tok@nd\endcsname{\def\PY@tc##1{\textcolor[rgb]{0.67,0.13,1.00}{##1}}}
\expandafter\def\csname PY@tok@s\endcsname{\def\PY@tc##1{\textcolor[rgb]{0.73,0.13,0.13}{##1}}}
\expandafter\def\csname PY@tok@sd\endcsname{\let\PY@it=\textit\def\PY@tc##1{\textcolor[rgb]{0.73,0.13,0.13}{##1}}}
\expandafter\def\csname PY@tok@si\endcsname{\let\PY@bf=\textbf\def\PY@tc##1{\textcolor[rgb]{0.73,0.40,0.53}{##1}}}
\expandafter\def\csname PY@tok@se\endcsname{\let\PY@bf=\textbf\def\PY@tc##1{\textcolor[rgb]{0.73,0.40,0.13}{##1}}}
\expandafter\def\csname PY@tok@sr\endcsname{\def\PY@tc##1{\textcolor[rgb]{0.73,0.40,0.53}{##1}}}
\expandafter\def\csname PY@tok@ss\endcsname{\def\PY@tc##1{\textcolor[rgb]{0.10,0.09,0.49}{##1}}}
\expandafter\def\csname PY@tok@sx\endcsname{\def\PY@tc##1{\textcolor[rgb]{0.00,0.50,0.00}{##1}}}
\expandafter\def\csname PY@tok@m\endcsname{\def\PY@tc##1{\textcolor[rgb]{0.40,0.40,0.40}{##1}}}
\expandafter\def\csname PY@tok@gh\endcsname{\let\PY@bf=\textbf\def\PY@tc##1{\textcolor[rgb]{0.00,0.00,0.50}{##1}}}
\expandafter\def\csname PY@tok@gu\endcsname{\let\PY@bf=\textbf\def\PY@tc##1{\textcolor[rgb]{0.50,0.00,0.50}{##1}}}
\expandafter\def\csname PY@tok@gd\endcsname{\def\PY@tc##1{\textcolor[rgb]{0.63,0.00,0.00}{##1}}}
\expandafter\def\csname PY@tok@gi\endcsname{\def\PY@tc##1{\textcolor[rgb]{0.00,0.63,0.00}{##1}}}
\expandafter\def\csname PY@tok@gr\endcsname{\def\PY@tc##1{\textcolor[rgb]{1.00,0.00,0.00}{##1}}}
\expandafter\def\csname PY@tok@ge\endcsname{\let\PY@it=\textit}
\expandafter\def\csname PY@tok@gs\endcsname{\let\PY@bf=\textbf}
\expandafter\def\csname PY@tok@gp\endcsname{\let\PY@bf=\textbf\def\PY@tc##1{\textcolor[rgb]{0.00,0.00,0.50}{##1}}}
\expandafter\def\csname PY@tok@go\endcsname{\def\PY@tc##1{\textcolor[rgb]{0.53,0.53,0.53}{##1}}}
\expandafter\def\csname PY@tok@gt\endcsname{\def\PY@tc##1{\textcolor[rgb]{0.00,0.27,0.87}{##1}}}
\expandafter\def\csname PY@tok@err\endcsname{\def\PY@bc##1{\setlength{\fboxsep}{0pt}\fcolorbox[rgb]{1.00,0.00,0.00}{1,1,1}{\strut ##1}}}
\expandafter\def\csname PY@tok@kc\endcsname{\let\PY@bf=\textbf\def\PY@tc##1{\textcolor[rgb]{0.00,0.50,0.00}{##1}}}
\expandafter\def\csname PY@tok@kd\endcsname{\let\PY@bf=\textbf\def\PY@tc##1{\textcolor[rgb]{0.00,0.50,0.00}{##1}}}
\expandafter\def\csname PY@tok@kn\endcsname{\let\PY@bf=\textbf\def\PY@tc##1{\textcolor[rgb]{0.00,0.50,0.00}{##1}}}
\expandafter\def\csname PY@tok@kr\endcsname{\let\PY@bf=\textbf\def\PY@tc##1{\textcolor[rgb]{0.00,0.50,0.00}{##1}}}
\expandafter\def\csname PY@tok@bp\endcsname{\def\PY@tc##1{\textcolor[rgb]{0.00,0.50,0.00}{##1}}}
\expandafter\def\csname PY@tok@fm\endcsname{\def\PY@tc##1{\textcolor[rgb]{0.00,0.00,1.00}{##1}}}
\expandafter\def\csname PY@tok@vc\endcsname{\def\PY@tc##1{\textcolor[rgb]{0.10,0.09,0.49}{##1}}}
\expandafter\def\csname PY@tok@vg\endcsname{\def\PY@tc##1{\textcolor[rgb]{0.10,0.09,0.49}{##1}}}
\expandafter\def\csname PY@tok@vi\endcsname{\def\PY@tc##1{\textcolor[rgb]{0.10,0.09,0.49}{##1}}}
\expandafter\def\csname PY@tok@vm\endcsname{\def\PY@tc##1{\textcolor[rgb]{0.10,0.09,0.49}{##1}}}
\expandafter\def\csname PY@tok@sa\endcsname{\def\PY@tc##1{\textcolor[rgb]{0.73,0.13,0.13}{##1}}}
\expandafter\def\csname PY@tok@sb\endcsname{\def\PY@tc##1{\textcolor[rgb]{0.73,0.13,0.13}{##1}}}
\expandafter\def\csname PY@tok@sc\endcsname{\def\PY@tc##1{\textcolor[rgb]{0.73,0.13,0.13}{##1}}}
\expandafter\def\csname PY@tok@dl\endcsname{\def\PY@tc##1{\textcolor[rgb]{0.73,0.13,0.13}{##1}}}
\expandafter\def\csname PY@tok@s2\endcsname{\def\PY@tc##1{\textcolor[rgb]{0.73,0.13,0.13}{##1}}}
\expandafter\def\csname PY@tok@sh\endcsname{\def\PY@tc##1{\textcolor[rgb]{0.73,0.13,0.13}{##1}}}
\expandafter\def\csname PY@tok@s1\endcsname{\def\PY@tc##1{\textcolor[rgb]{0.73,0.13,0.13}{##1}}}
\expandafter\def\csname PY@tok@mb\endcsname{\def\PY@tc##1{\textcolor[rgb]{0.40,0.40,0.40}{##1}}}
\expandafter\def\csname PY@tok@mf\endcsname{\def\PY@tc##1{\textcolor[rgb]{0.40,0.40,0.40}{##1}}}
\expandafter\def\csname PY@tok@mh\endcsname{\def\PY@tc##1{\textcolor[rgb]{0.40,0.40,0.40}{##1}}}
\expandafter\def\csname PY@tok@mi\endcsname{\def\PY@tc##1{\textcolor[rgb]{0.40,0.40,0.40}{##1}}}
\expandafter\def\csname PY@tok@il\endcsname{\def\PY@tc##1{\textcolor[rgb]{0.40,0.40,0.40}{##1}}}
\expandafter\def\csname PY@tok@mo\endcsname{\def\PY@tc##1{\textcolor[rgb]{0.40,0.40,0.40}{##1}}}
\expandafter\def\csname PY@tok@ch\endcsname{\let\PY@it=\textit\def\PY@tc##1{\textcolor[rgb]{0.25,0.50,0.50}{##1}}}
\expandafter\def\csname PY@tok@cm\endcsname{\let\PY@it=\textit\def\PY@tc##1{\textcolor[rgb]{0.25,0.50,0.50}{##1}}}
\expandafter\def\csname PY@tok@cpf\endcsname{\let\PY@it=\textit\def\PY@tc##1{\textcolor[rgb]{0.25,0.50,0.50}{##1}}}
\expandafter\def\csname PY@tok@c1\endcsname{\let\PY@it=\textit\def\PY@tc##1{\textcolor[rgb]{0.25,0.50,0.50}{##1}}}
\expandafter\def\csname PY@tok@cs\endcsname{\let\PY@it=\textit\def\PY@tc##1{\textcolor[rgb]{0.25,0.50,0.50}{##1}}}

\def\PYZbs{\char`\\}
\def\PYZus{\char`\_}
\def\PYZob{\char`\{}
\def\PYZcb{\char`\}}
\def\PYZca{\char`\^}
\def\PYZam{\char`\&}
\def\PYZlt{\char`\<}
\def\PYZgt{\char`\>}
\def\PYZsh{\char`\#}
\def\PYZpc{\char`\%}
\def\PYZdl{\char`\$}
\def\PYZhy{\char`\-}
\def\PYZsq{\char`\'}
\def\PYZdq{\char`\"}
\def\PYZti{\char`\~}
% for compatibility with earlier versions
\def\PYZat{@}
\def\PYZlb{[}
\def\PYZrb{]}
\makeatother


    % For linebreaks inside Verbatim environment from package fancyvrb. 
    \makeatletter
        \newbox\Wrappedcontinuationbox 
        \newbox\Wrappedvisiblespacebox 
        \newcommand*\Wrappedvisiblespace {\textcolor{red}{\textvisiblespace}} 
        \newcommand*\Wrappedcontinuationsymbol {\textcolor{red}{\llap{\tiny$\m@th\hookrightarrow$}}} 
        \newcommand*\Wrappedcontinuationindent {3ex } 
        \newcommand*\Wrappedafterbreak {\kern\Wrappedcontinuationindent\copy\Wrappedcontinuationbox} 
        % Take advantage of the already applied Pygments mark-up to insert 
        % potential linebreaks for TeX processing. 
        %        {, <, #, %, $, ' and ": go to next line. 
        %        _, }, ^, &, >, - and ~: stay at end of broken line. 
        % Use of \textquotesingle for straight quote. 
        \newcommand*\Wrappedbreaksatspecials {% 
            \def\PYGZus{\discretionary{\char`\_}{\Wrappedafterbreak}{\char`\_}}% 
            \def\PYGZob{\discretionary{}{\Wrappedafterbreak\char`\{}{\char`\{}}% 
            \def\PYGZcb{\discretionary{\char`\}}{\Wrappedafterbreak}{\char`\}}}% 
            \def\PYGZca{\discretionary{\char`\^}{\Wrappedafterbreak}{\char`\^}}% 
            \def\PYGZam{\discretionary{\char`\&}{\Wrappedafterbreak}{\char`\&}}% 
            \def\PYGZlt{\discretionary{}{\Wrappedafterbreak\char`\<}{\char`\<}}% 
            \def\PYGZgt{\discretionary{\char`\>}{\Wrappedafterbreak}{\char`\>}}% 
            \def\PYGZsh{\discretionary{}{\Wrappedafterbreak\char`\#}{\char`\#}}% 
            \def\PYGZpc{\discretionary{}{\Wrappedafterbreak\char`\%}{\char`\%}}% 
            \def\PYGZdl{\discretionary{}{\Wrappedafterbreak\char`\$}{\char`\$}}% 
            \def\PYGZhy{\discretionary{\char`\-}{\Wrappedafterbreak}{\char`\-}}% 
            \def\PYGZsq{\discretionary{}{\Wrappedafterbreak\textquotesingle}{\textquotesingle}}% 
            \def\PYGZdq{\discretionary{}{\Wrappedafterbreak\char`\"}{\char`\"}}% 
            \def\PYGZti{\discretionary{\char`\~}{\Wrappedafterbreak}{\char`\~}}% 
        } 
        % Some characters . , ; ? ! / are not pygmentized. 
        % This macro makes them "active" and they will insert potential linebreaks 
        \newcommand*\Wrappedbreaksatpunct {% 
            \lccode`\~`\.\lowercase{\def~}{\discretionary{\hbox{\char`\.}}{\Wrappedafterbreak}{\hbox{\char`\.}}}% 
            \lccode`\~`\,\lowercase{\def~}{\discretionary{\hbox{\char`\,}}{\Wrappedafterbreak}{\hbox{\char`\,}}}% 
            \lccode`\~`\;\lowercase{\def~}{\discretionary{\hbox{\char`\;}}{\Wrappedafterbreak}{\hbox{\char`\;}}}% 
            \lccode`\~`\:\lowercase{\def~}{\discretionary{\hbox{\char`\:}}{\Wrappedafterbreak}{\hbox{\char`\:}}}% 
            \lccode`\~`\?\lowercase{\def~}{\discretionary{\hbox{\char`\?}}{\Wrappedafterbreak}{\hbox{\char`\?}}}% 
            \lccode`\~`\!\lowercase{\def~}{\discretionary{\hbox{\char`\!}}{\Wrappedafterbreak}{\hbox{\char`\!}}}% 
            \lccode`\~`\/\lowercase{\def~}{\discretionary{\hbox{\char`\/}}{\Wrappedafterbreak}{\hbox{\char`\/}}}% 
            \catcode`\.\active
            \catcode`\,\active 
            \catcode`\;\active
            \catcode`\:\active
            \catcode`\?\active
            \catcode`\!\active
            \catcode`\/\active 
            \lccode`\~`\~ 	
        }
    \makeatother

    \let\OriginalVerbatim=\Verbatim
    \makeatletter
    \renewcommand{\Verbatim}[1][1]{%
        %\parskip\z@skip
        \sbox\Wrappedcontinuationbox {\Wrappedcontinuationsymbol}%
        \sbox\Wrappedvisiblespacebox {\FV@SetupFont\Wrappedvisiblespace}%
        \def\FancyVerbFormatLine ##1{\hsize\linewidth
            \vtop{\raggedright\hyphenpenalty\z@\exhyphenpenalty\z@
                \doublehyphendemerits\z@\finalhyphendemerits\z@
                \strut ##1\strut}%
        }%
        % If the linebreak is at a space, the latter will be displayed as visible
        % space at end of first line, and a continuation symbol starts next line.
        % Stretch/shrink are however usually zero for typewriter font.
        \def\FV@Space {%
            \nobreak\hskip\z@ plus\fontdimen3\font minus\fontdimen4\font
            \discretionary{\copy\Wrappedvisiblespacebox}{\Wrappedafterbreak}
            {\kern\fontdimen2\font}%
        }%
        
        % Allow breaks at special characters using \PYG... macros.
        \Wrappedbreaksatspecials
        % Breaks at punctuation characters . , ; ? ! and / need catcode=\active 	
        \OriginalVerbatim[#1,codes*=\Wrappedbreaksatpunct]%
    }
    \makeatother

    % Exact colors from NB
    \definecolor{incolor}{HTML}{303F9F}
    \definecolor{outcolor}{HTML}{D84315}
    \definecolor{cellborder}{HTML}{CFCFCF}
    \definecolor{cellbackground}{HTML}{F7F7F7}
    
    % prompt
    \makeatletter
    \newcommand{\boxspacing}{\kern\kvtcb@left@rule\kern\kvtcb@boxsep}
    \makeatother
    \newcommand{\prompt}[4]{
        \ttfamily\llap{{\color{#2}[#3]:\hspace{3pt}#4}}\vspace{-\baselineskip}
    }
    

    
    % Prevent overflowing lines due to hard-to-break entities
    \sloppy 
    % Setup hyperref package
    \hypersetup{
      breaklinks=true,  % so long urls are correctly broken across lines
      colorlinks=true,
      urlcolor=urlcolor,
      linkcolor=linkcolor,
      citecolor=citecolor,
      }
    % Slightly bigger margins than the latex defaults
    
    \geometry{verbose,tmargin=1in,bmargin=1in,lmargin=1in,rmargin=1in}
    
    

\begin{document}
    
    \maketitle
    
    

    
    \begin{tcolorbox}[breakable, size=fbox, boxrule=1pt, pad at break*=1mm,colback=cellbackground, colframe=cellborder]
\prompt{In}{incolor}{1}{\boxspacing}
\begin{Verbatim}[commandchars=\\\{\}]
\PY{k+kn}{import} \PY{n+nn}{numpy} \PY{k}{as} \PY{n+nn}{np}
\end{Verbatim}
\end{tcolorbox}

    \hypertarget{create-numpy-array---i}{%
\section{Create Numpy Array - I}\label{create-numpy-array---i}}

    \begin{tcolorbox}[breakable, size=fbox, boxrule=1pt, pad at break*=1mm,colback=cellbackground, colframe=cellborder]
\prompt{In}{incolor}{3}{\boxspacing}
\begin{Verbatim}[commandchars=\\\{\}]
\PY{n}{arr\PYZus{}1d} \PY{o}{=} \PY{n}{np}\PY{o}{.}\PY{n}{array}\PY{p}{(}\PY{p}{[}\PY{l+m+mi}{4}\PY{p}{,}\PY{l+m+mi}{2}\PY{p}{,}\PY{l+m+mi}{3}\PY{p}{]}\PY{p}{)}
\end{Verbatim}
\end{tcolorbox}

    \begin{tcolorbox}[breakable, size=fbox, boxrule=1pt, pad at break*=1mm,colback=cellbackground, colframe=cellborder]
\prompt{In}{incolor}{4}{\boxspacing}
\begin{Verbatim}[commandchars=\\\{\}]
\PY{n+nb}{type}\PY{p}{(}\PY{n}{arr\PYZus{}1d}\PY{p}{)}
\end{Verbatim}
\end{tcolorbox}

            \begin{tcolorbox}[breakable, size=fbox, boxrule=.5pt, pad at break*=1mm, opacityfill=0]
\prompt{Out}{outcolor}{4}{\boxspacing}
\begin{Verbatim}[commandchars=\\\{\}]
numpy.ndarray
\end{Verbatim}
\end{tcolorbox}
        
    \begin{tcolorbox}[breakable, size=fbox, boxrule=1pt, pad at break*=1mm,colback=cellbackground, colframe=cellborder]
\prompt{In}{incolor}{6}{\boxspacing}
\begin{Verbatim}[commandchars=\\\{\}]
\PY{n}{arr\PYZus{}2d} \PY{o}{=} \PY{n}{np}\PY{o}{.}\PY{n}{array}\PY{p}{(}\PY{p}{[}\PY{p}{[}\PY{l+m+mi}{3}\PY{p}{,}\PY{l+m+mi}{1}\PY{p}{,}\PY{l+m+mi}{2}\PY{p}{]}\PY{p}{,} \PY{p}{[}\PY{l+m+mi}{5}\PY{p}{,}\PY{l+m+mi}{4}\PY{p}{,}\PY{l+m+mi}{6}\PY{p}{]}\PY{p}{]}\PY{p}{)}
\PY{n}{arr\PYZus{}2d}
\end{Verbatim}
\end{tcolorbox}

            \begin{tcolorbox}[breakable, size=fbox, boxrule=.5pt, pad at break*=1mm, opacityfill=0]
\prompt{Out}{outcolor}{6}{\boxspacing}
\begin{Verbatim}[commandchars=\\\{\}]
array([[3, 1, 2],
       [5, 4, 6]])
\end{Verbatim}
\end{tcolorbox}
        
    \begin{tcolorbox}[breakable, size=fbox, boxrule=1pt, pad at break*=1mm,colback=cellbackground, colframe=cellborder]
\prompt{In}{incolor}{23}{\boxspacing}
\begin{Verbatim}[commandchars=\\\{\}]
\PY{n}{arr\PYZus{}3d} \PY{o}{=} \PY{n}{np}\PY{o}{.}\PY{n}{array}\PY{p}{(}\PY{p}{[}
    \PY{p}{[}
        \PY{p}{[}\PY{l+m+mi}{2}\PY{p}{,}\PY{l+m+mi}{3}\PY{p}{]}\PY{p}{,} \PY{p}{[}\PY{l+m+mi}{4}\PY{p}{,}\PY{l+m+mi}{1}\PY{p}{]}\PY{p}{,} \PY{p}{[}\PY{l+m+mi}{3}\PY{p}{,}\PY{l+m+mi}{1}\PY{p}{]}
    \PY{p}{]}\PY{p}{,} 
    \PY{p}{[}
        \PY{p}{[}\PY{l+m+mi}{33}\PY{p}{,}\PY{l+m+mi}{1}\PY{p}{]}\PY{p}{,} \PY{p}{[}\PY{l+m+mi}{9}\PY{p}{,}\PY{l+m+mi}{5}\PY{p}{]}\PY{p}{,} \PY{p}{[}\PY{l+m+mi}{4}\PY{p}{,}\PY{l+m+mi}{2}\PY{p}{]}
    \PY{p}{]}\PY{p}{,} 
    \PY{p}{[}
        \PY{p}{[}\PY{l+m+mi}{2}\PY{p}{,}\PY{l+m+mi}{3}\PY{p}{]}\PY{p}{,} \PY{p}{[}\PY{l+m+mi}{4}\PY{p}{,}\PY{l+m+mi}{1}\PY{p}{]}\PY{p}{,} \PY{p}{[}\PY{l+m+mi}{2}\PY{p}{,}\PY{l+m+mi}{1}\PY{p}{]}
    \PY{p}{]}\PY{p}{,} 
    \PY{p}{[}
        \PY{p}{[}\PY{l+m+mi}{33}\PY{p}{,}\PY{l+m+mi}{1}\PY{p}{]}\PY{p}{,} \PY{p}{[}\PY{l+m+mi}{9}\PY{p}{,}\PY{l+m+mi}{5}\PY{p}{]}\PY{p}{,} \PY{p}{[}\PY{l+m+mi}{3}\PY{p}{,}\PY{l+m+mi}{1}\PY{p}{]}
    \PY{p}{]}
\PY{p}{]}\PY{p}{)}
\PY{n}{arr\PYZus{}3d}
\end{Verbatim}
\end{tcolorbox}

            \begin{tcolorbox}[breakable, size=fbox, boxrule=.5pt, pad at break*=1mm, opacityfill=0]
\prompt{Out}{outcolor}{23}{\boxspacing}
\begin{Verbatim}[commandchars=\\\{\}]
array([[[ 2,  3],
        [ 4,  1],
        [ 3,  1]],

       [[33,  1],
        [ 9,  5],
        [ 4,  2]],

       [[ 2,  3],
        [ 4,  1],
        [ 2,  1]],

       [[33,  1],
        [ 9,  5],
        [ 3,  1]]])
\end{Verbatim}
\end{tcolorbox}
        
    \hypertarget{array-atrributes}{%
\section{Array Atrributes}\label{array-atrributes}}

    \begin{tcolorbox}[breakable, size=fbox, boxrule=1pt, pad at break*=1mm,colback=cellbackground, colframe=cellborder]
\prompt{In}{incolor}{24}{\boxspacing}
\begin{Verbatim}[commandchars=\\\{\}]
\PY{n+nb}{print}\PY{p}{(}\PY{n}{arr\PYZus{}1d}\PY{o}{.}\PY{n}{shape}\PY{p}{)}
\PY{n+nb}{print}\PY{p}{(}\PY{n}{arr\PYZus{}2d}\PY{o}{.}\PY{n}{shape}\PY{p}{)}
\PY{n+nb}{print}\PY{p}{(}\PY{n}{arr\PYZus{}3d}\PY{o}{.}\PY{n}{shape}\PY{p}{)}
\end{Verbatim}
\end{tcolorbox}

    \begin{Verbatim}[commandchars=\\\{\}]
(3,)
(2, 3)
(4, 3, 2)
    \end{Verbatim}

    \begin{tcolorbox}[breakable, size=fbox, boxrule=1pt, pad at break*=1mm,colback=cellbackground, colframe=cellborder]
\prompt{In}{incolor}{35}{\boxspacing}
\begin{Verbatim}[commandchars=\\\{\}]
\PY{c+c1}{\PYZsh{} The first argument of shape}
\PY{n+nb}{print}\PY{p}{(}\PY{n+nb}{len}\PY{p}{(}\PY{n}{arr\PYZus{}1d}\PY{p}{)}\PY{p}{,} \PY{n}{np}\PY{o}{.}\PY{n}{size}\PY{p}{(}\PY{n}{arr\PYZus{}1d}\PY{p}{,} \PY{l+m+mi}{0}\PY{p}{)}\PY{p}{)}
\PY{n+nb}{print}\PY{p}{(}\PY{n+nb}{len}\PY{p}{(}\PY{n}{arr\PYZus{}2d}\PY{p}{)}\PY{p}{,} \PY{n}{np}\PY{o}{.}\PY{n}{size}\PY{p}{(}\PY{n}{arr\PYZus{}2d}\PY{p}{,} \PY{l+m+mi}{0}\PY{p}{)}\PY{p}{)}
\PY{n+nb}{print}\PY{p}{(}\PY{n+nb}{len}\PY{p}{(}\PY{n}{arr\PYZus{}3d}\PY{p}{)}\PY{p}{,} \PY{n}{np}\PY{o}{.}\PY{n}{size}\PY{p}{(}\PY{n}{arr\PYZus{}3d}\PY{p}{,} \PY{l+m+mi}{0}\PY{p}{)}\PY{p}{)}
\end{Verbatim}
\end{tcolorbox}

    \begin{Verbatim}[commandchars=\\\{\}]
3 3
2 2
4 4
    \end{Verbatim}

    \begin{tcolorbox}[breakable, size=fbox, boxrule=1pt, pad at break*=1mm,colback=cellbackground, colframe=cellborder]
\prompt{In}{incolor}{38}{\boxspacing}
\begin{Verbatim}[commandchars=\\\{\}]
\PY{c+c1}{\PYZsh{} The number of elements along a given axis.}
\PY{c+c1}{\PYZsh{} If axis = None, return number of element of array.}
\PY{n+nb}{print}\PY{p}{(}\PY{n}{arr\PYZus{}1d}\PY{o}{.}\PY{n}{size}\PY{p}{,} \PY{n}{np}\PY{o}{.}\PY{n}{size}\PY{p}{(}\PY{n}{arr\PYZus{}1d}\PY{p}{)}\PY{p}{)}
\PY{n+nb}{print}\PY{p}{(}\PY{n}{arr\PYZus{}2d}\PY{o}{.}\PY{n}{size}\PY{p}{,} \PY{n}{np}\PY{o}{.}\PY{n}{size}\PY{p}{(}\PY{n}{arr\PYZus{}2d}\PY{p}{)}\PY{p}{)}
\PY{n+nb}{print}\PY{p}{(}\PY{n}{arr\PYZus{}3d}\PY{o}{.}\PY{n}{size}\PY{p}{,} \PY{n}{np}\PY{o}{.}\PY{n}{size}\PY{p}{(}\PY{n}{arrr\PYZus{}3d}\PY{p}{)}\PY{p}{)}
\end{Verbatim}
\end{tcolorbox}

    \begin{Verbatim}[commandchars=\\\{\}]
3 3
6 6
24 24
    \end{Verbatim}

    \begin{tcolorbox}[breakable, size=fbox, boxrule=1pt, pad at break*=1mm,colback=cellbackground, colframe=cellborder]
\prompt{In}{incolor}{47}{\boxspacing}
\begin{Verbatim}[commandchars=\\\{\}]
\PY{c+c1}{\PYZsh{} Return data type object}
\PY{n+nb}{print}\PY{p}{(}\PY{n}{arr\PYZus{}1d}\PY{o}{.}\PY{n}{dtype}\PY{p}{)}
\PY{n+nb}{print}\PY{p}{(}\PY{n}{np}\PY{o}{.}\PY{n}{array}\PY{p}{(}\PY{p}{[}\PY{l+s+s2}{\PYZdq{}}\PY{l+s+s2}{Hello}\PY{l+s+s2}{\PYZdq{}}\PY{p}{,} \PY{l+s+s2}{\PYZdq{}}\PY{l+s+s2}{World}\PY{l+s+s2}{\PYZdq{}}\PY{p}{]}\PY{p}{)}\PY{o}{.}\PY{n}{dtype}\PY{p}{)}
\PY{n+nb}{print}\PY{p}{(}\PY{n}{np}\PY{o}{.}\PY{n}{array}\PY{p}{(}\PY{p}{[}\PY{l+m+mi}{1}\PY{p}{,} \PY{l+m+mi}{2}\PY{p}{,} \PY{l+m+mi}{3}\PY{p}{]}\PY{p}{,} \PY{n}{dtype}\PY{o}{=}\PY{l+s+s2}{\PYZdq{}}\PY{l+s+s2}{f}\PY{l+s+s2}{\PYZdq{}}\PY{p}{)}\PY{o}{.}\PY{n}{dtype}\PY{p}{)}
\end{Verbatim}
\end{tcolorbox}

    \begin{Verbatim}[commandchars=\\\{\}]
int32
<U5
float32
    \end{Verbatim}

    \begin{tcolorbox}[breakable, size=fbox, boxrule=1pt, pad at break*=1mm,colback=cellbackground, colframe=cellborder]
\prompt{In}{incolor}{51}{\boxspacing}
\begin{Verbatim}[commandchars=\\\{\}]
\PY{c+c1}{\PYZsh{} Return dimension of narray}
\PY{n+nb}{print}\PY{p}{(}\PY{n}{arr\PYZus{}1d}\PY{o}{.}\PY{n}{ndim}\PY{p}{)}
\PY{n+nb}{print}\PY{p}{(}\PY{n}{arr\PYZus{}2d}\PY{o}{.}\PY{n}{ndim}\PY{p}{)}
\PY{n+nb}{print}\PY{p}{(}\PY{n}{arr\PYZus{}3d}\PY{o}{.}\PY{n}{ndim}\PY{p}{)}
\end{Verbatim}
\end{tcolorbox}

    \begin{Verbatim}[commandchars=\\\{\}]
1
2
3
    \end{Verbatim}

    \hypertarget{create-numpy-array---ii}{%
\section{Create Numpy Array - II}\label{create-numpy-array---ii}}

    \begin{tcolorbox}[breakable, size=fbox, boxrule=1pt, pad at break*=1mm,colback=cellbackground, colframe=cellborder]
\prompt{In}{incolor}{55}{\boxspacing}
\begin{Verbatim}[commandchars=\\\{\}]
\PY{c+c1}{\PYZsh{} arange([start,] stop[, step,], dtype=None)}
\PY{n}{even\PYZus{}arr} \PY{o}{=} \PY{n}{np}\PY{o}{.}\PY{n}{arange}\PY{p}{(}\PY{l+m+mi}{1}\PY{p}{,} \PY{l+m+mi}{10}\PY{p}{,} \PY{l+m+mi}{2}\PY{p}{)}
\PY{n+nb}{print}\PY{p}{(}\PY{n}{even\PYZus{}arr}\PY{p}{)}
\end{Verbatim}
\end{tcolorbox}

    \begin{Verbatim}[commandchars=\\\{\}]
[1 3 5 7 9]
    \end{Verbatim}

    \begin{tcolorbox}[breakable, size=fbox, boxrule=1pt, pad at break*=1mm,colback=cellbackground, colframe=cellborder]
\prompt{In}{incolor}{65}{\boxspacing}
\begin{Verbatim}[commandchars=\\\{\}]
\PY{c+c1}{\PYZsh{} np.linspace(start, stop, num=50, endpoint=True,retstep=False,dtype=None,axis=0,)}
\PY{c+c1}{\PYZsh{} If endpoint = True}
\PY{n+nb}{print}\PY{p}{(}\PY{n}{np}\PY{o}{.}\PY{n}{linspace}\PY{p}{(}\PY{l+m+mi}{1}\PY{p}{,} \PY{l+m+mi}{99}\PY{p}{,} \PY{l+m+mi}{50}\PY{p}{)}\PY{p}{)}
\PY{c+c1}{\PYZsh{} If endpoint = False}
\PY{n+nb}{print}\PY{p}{(}\PY{n}{np}\PY{o}{.}\PY{n}{linspace}\PY{p}{(}\PY{l+m+mi}{1}\PY{p}{,} \PY{l+m+mi}{101}\PY{p}{,} \PY{l+m+mi}{50}\PY{p}{,} \PY{n}{endpoint}\PY{o}{=}\PY{k+kc}{False}\PY{p}{)}\PY{p}{)}
\end{Verbatim}
\end{tcolorbox}

    \begin{Verbatim}[commandchars=\\\{\}]
[ 1.  3.  5.  7.  9. 11. 13. 15. 17. 19. 21. 23. 25. 27. 29. 31. 33. 35.
 37. 39. 41. 43. 45. 47. 49. 51. 53. 55. 57. 59. 61. 63. 65. 67. 69. 71.
 73. 75. 77. 79. 81. 83. 85. 87. 89. 91. 93. 95. 97. 99.]
[ 1.  3.  5.  7.  9. 11. 13. 15. 17. 19. 21. 23. 25. 27. 29. 31. 33. 35.
 37. 39. 41. 43. 45. 47. 49. 51. 53. 55. 57. 59. 61. 63. 65. 67. 69. 71.
 73. 75. 77. 79. 81. 83. 85. 87. 89. 91. 93. 95. 97. 99.]
    \end{Verbatim}

    \begin{tcolorbox}[breakable, size=fbox, boxrule=1pt, pad at break*=1mm,colback=cellbackground, colframe=cellborder]
\prompt{In}{incolor}{76}{\boxspacing}
\begin{Verbatim}[commandchars=\\\{\}]
\PY{c+c1}{\PYZsh{} Return numbers spaced evenly on a log scale.}
\PY{n+nb}{print}\PY{p}{(}\PY{n}{np}\PY{o}{.}\PY{n}{logspace}\PY{p}{(}\PY{l+m+mi}{1}\PY{p}{,} \PY{l+m+mi}{5}\PY{p}{,} \PY{l+m+mi}{10}\PY{p}{)}\PY{p}{)}
\PY{n}{np}\PY{o}{.}\PY{n}{logspace}
\end{Verbatim}
\end{tcolorbox}

    \begin{Verbatim}[commandchars=\\\{\}]
[1.00000000e+01 2.78255940e+01 7.74263683e+01 2.15443469e+02
 5.99484250e+02 1.66810054e+03 4.64158883e+03 1.29154967e+04
 3.59381366e+04 1.00000000e+05]
    \end{Verbatim}

            \begin{tcolorbox}[breakable, size=fbox, boxrule=.5pt, pad at break*=1mm, opacityfill=0]
\prompt{Out}{outcolor}{76}{\boxspacing}
\begin{Verbatim}[commandchars=\\\{\}]
<function numpy.logspace(start, stop, num=50, endpoint=True, base=10.0,
dtype=None, axis=0)>
\end{Verbatim}
\end{tcolorbox}
        
    \begin{tcolorbox}[breakable, size=fbox, boxrule=1pt, pad at break*=1mm,colback=cellbackground, colframe=cellborder]
\prompt{In}{incolor}{84}{\boxspacing}
\begin{Verbatim}[commandchars=\\\{\}]
\PY{c+c1}{\PYZsh{} zeros(shape, dtype=float, order=\PYZsq{}C\PYZsq{})}
\PY{c+c1}{\PYZsh{} Return a new array of given shape and type, filled with zeros.}
\PY{n+nb}{print}\PY{p}{(}\PY{n}{np}\PY{o}{.}\PY{n}{zeros}\PY{p}{(}\PY{p}{(}\PY{l+m+mi}{2}\PY{p}{,}\PY{l+m+mi}{3}\PY{p}{,}\PY{l+m+mi}{4}\PY{p}{)}\PY{p}{,} \PY{n+nb}{int}\PY{p}{)}\PY{p}{)}
\PY{c+c1}{\PYZsh{} ones}
\PY{n+nb}{print}\PY{p}{(}\PY{n}{np}\PY{o}{.}\PY{n}{ones}\PY{p}{(}\PY{p}{(}\PY{l+m+mi}{2}\PY{p}{,}\PY{l+m+mi}{3}\PY{p}{,}\PY{l+m+mi}{4}\PY{p}{)}\PY{p}{,} \PY{n+nb}{int}\PY{p}{)}\PY{p}{)}
\PY{c+c1}{\PYZsh{} full}
\PY{n+nb}{print}\PY{p}{(}\PY{n}{np}\PY{o}{.}\PY{n}{full}\PY{p}{(}\PY{p}{(}\PY{l+m+mi}{2}\PY{p}{,}\PY{l+m+mi}{3}\PY{p}{,}\PY{l+m+mi}{4}\PY{p}{)}\PY{p}{,} \PY{l+m+mi}{5}\PY{p}{)}\PY{p}{)}
\PY{c+c1}{\PYZsh{} eyes}
\PY{n+nb}{print}\PY{p}{(}\PY{n}{np}\PY{o}{.}\PY{n}{eye}\PY{p}{(}\PY{l+m+mi}{4}\PY{p}{)}\PY{p}{)}
\end{Verbatim}
\end{tcolorbox}

    \begin{Verbatim}[commandchars=\\\{\}]
[[[0 0 0 0]
  [0 0 0 0]
  [0 0 0 0]]

 [[0 0 0 0]
  [0 0 0 0]
  [0 0 0 0]]]
[[[1 1 1 1]
  [1 1 1 1]
  [1 1 1 1]]

 [[1 1 1 1]
  [1 1 1 1]
  [1 1 1 1]]]
[[[5 5 5 5]
  [5 5 5 5]
  [5 5 5 5]]

 [[5 5 5 5]
  [5 5 5 5]
  [5 5 5 5]]]
[[1. 0. 0. 0.]
 [0. 1. 0. 0.]
 [0. 0. 1. 0.]
 [0. 0. 0. 1.]]
    \end{Verbatim}

    \hypertarget{create-random-number-array}{%
\section{Create random number array}\label{create-random-number-array}}

    \begin{tcolorbox}[breakable, size=fbox, boxrule=1pt, pad at break*=1mm,colback=cellbackground, colframe=cellborder]
\prompt{In}{incolor}{91}{\boxspacing}
\begin{Verbatim}[commandchars=\\\{\}]
\PY{k+kn}{from} \PY{n+nn}{numpy} \PY{k+kn}{import} \PY{n}{random}
\end{Verbatim}
\end{tcolorbox}

    \begin{tcolorbox}[breakable, size=fbox, boxrule=1pt, pad at break*=1mm,colback=cellbackground, colframe=cellborder]
\prompt{In}{incolor}{95}{\boxspacing}
\begin{Verbatim}[commandchars=\\\{\}]
\PY{c+c1}{\PYZsh{} randint(low, high=None, size=None, dtype=int)}
\PY{n+nb}{print}\PY{p}{(}\PY{n}{random}\PY{o}{.}\PY{n}{randint}\PY{p}{(}\PY{l+m+mi}{1}\PY{p}{,} \PY{l+m+mi}{10}\PY{p}{,} \PY{p}{(}\PY{l+m+mi}{2}\PY{p}{,}\PY{l+m+mi}{3}\PY{p}{)}\PY{p}{)}\PY{p}{)}
\PY{c+c1}{\PYZsh{} rand : Random values in a given shape}
\PY{n+nb}{print}\PY{p}{(}\PY{n}{random}\PY{o}{.}\PY{n}{rand}\PY{p}{(}\PY{l+m+mi}{2}\PY{p}{,} \PY{l+m+mi}{3}\PY{p}{)}\PY{p}{)}
\end{Verbatim}
\end{tcolorbox}

    \begin{Verbatim}[commandchars=\\\{\}]
[[6 7 5]
 [3 8 2]]
[[0.48131667 0.42971598 0.15752261]
 [0.63592677 0.03942901 0.06625143]]
    \end{Verbatim}

    \begin{tcolorbox}[breakable, size=fbox, boxrule=1pt, pad at break*=1mm,colback=cellbackground, colframe=cellborder]
\prompt{In}{incolor}{111}{\boxspacing}
\begin{Verbatim}[commandchars=\\\{\}]
\PY{n}{arr} \PY{o}{=} \PY{n}{np}\PY{o}{.}\PY{n}{arange}\PY{p}{(}\PY{l+m+mi}{10}\PY{p}{)}
\PY{c+c1}{\PYZsh{} Shuffle:  makes changes to the original array.}
\PY{n}{random}\PY{o}{.}\PY{n}{shuffle}\PY{p}{(}\PY{n}{arr}\PY{p}{)}
\PY{n+nb}{print}\PY{p}{(}\PY{n}{arr}\PY{p}{)}
\PY{c+c1}{\PYZsh{} Permutation: returns a re\PYZhy{}arranged array (and leaves the original array un\PYZhy{}changed).}
\PY{n}{arr} \PY{o}{=} \PY{n}{np}\PY{o}{.}\PY{n}{arange}\PY{p}{(}\PY{l+m+mi}{10}\PY{p}{)}
\PY{n}{permutation\PYZus{}arr} \PY{o}{=} \PY{n}{random}\PY{o}{.}\PY{n}{permutation}\PY{p}{(}\PY{n}{arr}\PY{p}{)}
\PY{n+nb}{print}\PY{p}{(}\PY{l+s+s2}{\PYZdq{}}\PY{l+s+s2}{origin: }\PY{l+s+s2}{\PYZdq{}}\PY{p}{,} \PY{n}{arr}\PY{p}{)}
\PY{n+nb}{print}\PY{p}{(}\PY{l+s+s2}{\PYZdq{}}\PY{l+s+s2}{permutation: }\PY{l+s+s2}{\PYZdq{}}\PY{p}{,} \PY{n}{permutation\PYZus{}arr}\PY{p}{)}
\end{Verbatim}
\end{tcolorbox}

    \begin{Verbatim}[commandchars=\\\{\}]
[2 9 5 8 0 3 6 4 1 7]
origin:  [0 1 2 3 4 5 6 7 8 9]
permutation:  [9 2 3 4 5 7 6 1 8 0]
    \end{Verbatim}

    \hypertarget{numpy-indexing-slicing}{%
\section{Numpy Indexing \& Slicing}\label{numpy-indexing-slicing}}

    \hypertarget{array-broadcasting}{%
\section{Array Broadcasting}\label{array-broadcasting}}

    \begin{tcolorbox}[breakable, size=fbox, boxrule=1pt, pad at break*=1mm,colback=cellbackground, colframe=cellborder]
\prompt{In}{incolor}{121}{\boxspacing}
\begin{Verbatim}[commandchars=\\\{\}]
\PY{n}{arr1} \PY{o}{=} \PY{n}{np}\PY{o}{.}\PY{n}{arange}\PY{p}{(}\PY{l+m+mi}{1}\PY{p}{,} \PY{l+m+mi}{17}\PY{p}{)}\PY{o}{.}\PY{n}{reshape}\PY{p}{(}\PY{p}{(}\PY{l+m+mi}{4}\PY{p}{,}\PY{l+m+mi}{4}\PY{p}{)}\PY{p}{)}
\PY{n}{arr2} \PY{o}{=} \PY{n}{np}\PY{o}{.}\PY{n}{copy}\PY{p}{(}\PY{n}{arr1}\PY{p}{)}
\end{Verbatim}
\end{tcolorbox}

    \begin{tcolorbox}[breakable, size=fbox, boxrule=1pt, pad at break*=1mm,colback=cellbackground, colframe=cellborder]
\prompt{In}{incolor}{130}{\boxspacing}
\begin{Verbatim}[commandchars=\\\{\}]
\PY{c+c1}{\PYZsh{} Plus}
\PY{n+nb}{print}\PY{p}{(}\PY{n}{arr1} \PY{o}{+} \PY{n}{arr2}\PY{p}{)}
\PY{c+c1}{\PYZsh{} Minus}
\PY{n+nb}{print}\PY{p}{(}\PY{n}{arr1} \PY{o}{\PYZhy{}} \PY{n}{arr2}\PY{p}{)}
\PY{c+c1}{\PYZsh{} Multiple}
\PY{n+nb}{print}\PY{p}{(}\PY{n}{np}\PY{o}{.}\PY{n}{multiply}\PY{p}{(}\PY{n}{arr1}\PY{p}{,} \PY{n}{arr2}\PY{p}{)}\PY{p}{)}
\PY{n+nb}{print}\PY{p}{(}\PY{n}{arr1} \PY{o}{*} \PY{l+m+mi}{2}\PY{p}{)}
\PY{c+c1}{\PYZsh{} Division}
\PY{n+nb}{print}\PY{p}{(}\PY{n}{arr1} \PY{o}{/} \PY{l+m+mi}{2}\PY{p}{)}
\end{Verbatim}
\end{tcolorbox}

    \begin{Verbatim}[commandchars=\\\{\}]
[[ 2  4  6  8]
 [10 12 14 16]
 [18 20 22 24]
 [26 28 30 32]]
[[0 0 0 0]
 [0 0 0 0]
 [0 0 0 0]
 [0 0 0 0]]
[[  1   4   9  16]
 [ 25  36  49  64]
 [ 81 100 121 144]
 [169 196 225 256]]
[[ 2  4  6  8]
 [10 12 14 16]
 [18 20 22 24]
 [26 28 30 32]]
[[0.5 1.  1.5 2. ]
 [2.5 3.  3.5 4. ]
 [4.5 5.  5.5 6. ]
 [6.5 7.  7.5 8. ]]
    \end{Verbatim}

    \hypertarget{array-manipulation}{%
\section{Array manipulation}\label{array-manipulation}}

    \begin{tcolorbox}[breakable, size=fbox, boxrule=1pt, pad at break*=1mm,colback=cellbackground, colframe=cellborder]
\prompt{In}{incolor}{147}{\boxspacing}
\begin{Verbatim}[commandchars=\\\{\}]
\PY{n+nb}{print}\PY{p}{(}\PY{n}{arr1}\PY{o}{.}\PY{n}{reshape}\PY{p}{(}\PY{o}{\PYZhy{}}\PY{l+m+mi}{1}\PY{p}{)}\PY{p}{)}
\PY{c+c1}{\PYZsh{} Flatten return a copy}
\PY{n}{arr1}\PY{o}{.}\PY{n}{flatten}\PY{p}{(}\PY{p}{)}
\PY{c+c1}{\PYZsh{} Ravel return a view of the original array.}
\PY{n}{arr1}\PY{o}{.}\PY{n}{ravel}\PY{p}{(}\PY{p}{)}
\end{Verbatim}
\end{tcolorbox}

    \begin{Verbatim}[commandchars=\\\{\}]
[ 1  2  3  4  5  6  7  8  9 10 11 12 13 14 15 16]
    \end{Verbatim}

            \begin{tcolorbox}[breakable, size=fbox, boxrule=.5pt, pad at break*=1mm, opacityfill=0]
\prompt{Out}{outcolor}{147}{\boxspacing}
\begin{Verbatim}[commandchars=\\\{\}]
array([ 1,  2,  3,  4,  5,  6,  7,  8,  9, 10, 11, 12, 13, 14, 15, 16])
\end{Verbatim}
\end{tcolorbox}
        
    \begin{tcolorbox}[breakable, size=fbox, boxrule=1pt, pad at break*=1mm,colback=cellbackground, colframe=cellborder]
\prompt{In}{incolor}{150}{\boxspacing}
\begin{Verbatim}[commandchars=\\\{\}]
\PY{c+c1}{\PYZsh{} Transpose}
\PY{n}{arr1}\PY{o}{.}\PY{n}{transpose}\PY{p}{(}\PY{p}{)}
\PY{n}{arr1}\PY{o}{.}\PY{n}{T}
\end{Verbatim}
\end{tcolorbox}

            \begin{tcolorbox}[breakable, size=fbox, boxrule=.5pt, pad at break*=1mm, opacityfill=0]
\prompt{Out}{outcolor}{150}{\boxspacing}
\begin{Verbatim}[commandchars=\\\{\}]
array([[ 1,  5,  9, 13],
       [ 2,  6, 10, 14],
       [ 3,  7, 11, 15],
       [ 4,  8, 12, 16]])
\end{Verbatim}
\end{tcolorbox}
        
    \begin{tcolorbox}[breakable, size=fbox, boxrule=1pt, pad at break*=1mm,colback=cellbackground, colframe=cellborder]
\prompt{In}{incolor}{151}{\boxspacing}
\begin{Verbatim}[commandchars=\\\{\}]
\PY{n}{arr1}
\end{Verbatim}
\end{tcolorbox}

            \begin{tcolorbox}[breakable, size=fbox, boxrule=.5pt, pad at break*=1mm, opacityfill=0]
\prompt{Out}{outcolor}{151}{\boxspacing}
\begin{Verbatim}[commandchars=\\\{\}]
array([[ 1,  2,  3,  4],
       [ 5,  6,  7,  8],
       [ 9, 10, 11, 12],
       [13, 14, 15, 16]])
\end{Verbatim}
\end{tcolorbox}
        
    \begin{tcolorbox}[breakable, size=fbox, boxrule=1pt, pad at break*=1mm,colback=cellbackground, colframe=cellborder]
\prompt{In}{incolor}{168}{\boxspacing}
\begin{Verbatim}[commandchars=\\\{\}]
\PY{c+c1}{\PYZsh{} Join a sequence of arrays along an existing axis.}
\PY{c+c1}{\PYZsh{} concatenate((a1, a2, ...), axis=0, out=None)}
\PY{n}{vstack\PYZus{}arr} \PY{o}{=} \PY{n}{np}\PY{o}{.}\PY{n}{vstack}\PY{p}{(}\PY{p}{(}\PY{n}{arr1}\PY{p}{,} \PY{n}{arr2}\PY{p}{)}\PY{p}{)}
\PY{n+nb}{print}\PY{p}{(}\PY{n}{np}\PY{o}{.}\PY{n}{concatenate}\PY{p}{(}\PY{p}{(}\PY{n}{arr1}\PY{p}{,} \PY{n}{arr2}\PY{p}{)}\PY{p}{)}\PY{p}{)}
\PY{n+nb}{print}\PY{p}{(}\PY{n}{vstack\PYZus{}arr}\PY{p}{)}

\PY{n}{hstack\PYZus{}arr} \PY{o}{=} \PY{n}{np}\PY{o}{.}\PY{n}{hstack}\PY{p}{(}\PY{p}{(}\PY{n}{arr1}\PY{p}{,} \PY{n}{arr2}\PY{p}{)}\PY{p}{)}
\PY{n+nb}{print}\PY{p}{(}\PY{n}{np}\PY{o}{.}\PY{n}{concatenate}\PY{p}{(}\PY{p}{(}\PY{n}{arr1}\PY{p}{,} \PY{n}{arr2}\PY{p}{)}\PY{p}{,} \PY{n}{axis} \PY{o}{=} \PY{l+m+mi}{1}\PY{p}{)}\PY{p}{)}
\PY{n+nb}{print}\PY{p}{(}\PY{n}{hstack\PYZus{}arr}\PY{p}{)}
\end{Verbatim}
\end{tcolorbox}

    \begin{Verbatim}[commandchars=\\\{\}]
[[ 1  2  3  4]
 [ 5  6  7  8]
 [ 9 10 11 12]
 [13 14 15 16]
 [ 1  2  3  4]
 [ 5  6  7  8]
 [ 9 10 11 12]
 [13 14 15 16]]
[[ 1  2  3  4]
 [ 5  6  7  8]
 [ 9 10 11 12]
 [13 14 15 16]
 [ 1  2  3  4]
 [ 5  6  7  8]
 [ 9 10 11 12]
 [13 14 15 16]]
[[ 1  2  3  4  1  2  3  4]
 [ 5  6  7  8  5  6  7  8]
 [ 9 10 11 12  9 10 11 12]
 [13 14 15 16 13 14 15 16]]
[[ 1  2  3  4  1  2  3  4]
 [ 5  6  7  8  5  6  7  8]
 [ 9 10 11 12  9 10 11 12]
 [13 14 15 16 13 14 15 16]]
    \end{Verbatim}

    \begin{tcolorbox}[breakable, size=fbox, boxrule=1pt, pad at break*=1mm,colback=cellbackground, colframe=cellborder]
\prompt{In}{incolor}{173}{\boxspacing}
\begin{Verbatim}[commandchars=\\\{\}]
\PY{c+c1}{\PYZsh{} Split}
\PY{n+nb}{print}\PY{p}{(}\PY{n}{np}\PY{o}{.}\PY{n}{split}\PY{p}{(}\PY{n}{vstack\PYZus{}arr}\PY{p}{,} \PY{l+m+mi}{2}\PY{p}{)}\PY{p}{)}
\PY{n+nb}{print}\PY{p}{(}\PY{n}{np}\PY{o}{.}\PY{n}{vsplit}\PY{p}{(}\PY{n}{vstack\PYZus{}arr}\PY{p}{,} \PY{l+m+mi}{2}\PY{p}{)}\PY{p}{)}

\PY{n+nb}{print}\PY{p}{(}\PY{n}{np}\PY{o}{.}\PY{n}{split}\PY{p}{(}\PY{n}{hstack\PYZus{}arr}\PY{p}{,} \PY{l+m+mi}{2}\PY{p}{,} \PY{n}{axis} \PY{o}{=} \PY{l+m+mi}{1}\PY{p}{)}\PY{p}{)}
\PY{n+nb}{print}\PY{p}{(}\PY{n}{np}\PY{o}{.}\PY{n}{hsplit}\PY{p}{(}\PY{n}{hstack\PYZus{}arr}\PY{p}{,} \PY{l+m+mi}{2}\PY{p}{)}\PY{p}{)}
\end{Verbatim}
\end{tcolorbox}

    \begin{Verbatim}[commandchars=\\\{\}]
[array([[ 1,  2,  3,  4],
       [ 5,  6,  7,  8],
       [ 9, 10, 11, 12],
       [13, 14, 15, 16]]), array([[ 1,  2,  3,  4],
       [ 5,  6,  7,  8],
       [ 9, 10, 11, 12],
       [13, 14, 15, 16]])]
[array([[ 1,  2,  3,  4],
       [ 5,  6,  7,  8],
       [ 9, 10, 11, 12],
       [13, 14, 15, 16]]), array([[ 1,  2,  3,  4],
       [ 5,  6,  7,  8],
       [ 9, 10, 11, 12],
       [13, 14, 15, 16]])]
[array([[ 1,  2,  3,  4],
       [ 5,  6,  7,  8],
       [ 9, 10, 11, 12],
       [13, 14, 15, 16]]), array([[ 1,  2,  3,  4],
       [ 5,  6,  7,  8],
       [ 9, 10, 11, 12],
       [13, 14, 15, 16]])]
[array([[ 1,  2,  3,  4],
       [ 5,  6,  7,  8],
       [ 9, 10, 11, 12],
       [13, 14, 15, 16]]), array([[ 1,  2,  3,  4],
       [ 5,  6,  7,  8],
       [ 9, 10, 11, 12],
       [13, 14, 15, 16]])]
    \end{Verbatim}

    \hypertarget{binary-operations}{%
\section{Binary Operations}\label{binary-operations}}

    \hypertarget{math-operations}{%
\section{Math Operations}\label{math-operations}}

    \begin{tcolorbox}[breakable, size=fbox, boxrule=1pt, pad at break*=1mm,colback=cellbackground, colframe=cellborder]
\prompt{In}{incolor}{176}{\boxspacing}
\begin{Verbatim}[commandchars=\\\{\}]
\PY{n+nb}{print}\PY{p}{(}\PY{n}{np}\PY{o}{.}\PY{n}{add}\PY{p}{(}\PY{n}{arr1}\PY{p}{,} \PY{n}{arr2}\PY{p}{)}\PY{p}{)}
\PY{n+nb}{print}\PY{p}{(}\PY{n}{np}\PY{o}{.}\PY{n}{subtract}\PY{p}{(}\PY{n}{arr1}\PY{p}{,} \PY{n}{arr2}\PY{p}{)}\PY{p}{)}
\PY{n+nb}{print}\PY{p}{(}\PY{n}{np}\PY{o}{.}\PY{n}{multiply}\PY{p}{(}\PY{n}{arr1}\PY{p}{,} \PY{n}{arr2}\PY{p}{)}\PY{p}{)}
\PY{n+nb}{print}\PY{p}{(}\PY{n}{np}\PY{o}{.}\PY{n}{divide}\PY{p}{(}\PY{n}{arr1}\PY{p}{,} \PY{n}{arr2}\PY{p}{)}\PY{p}{)}
\end{Verbatim}
\end{tcolorbox}

    \begin{Verbatim}[commandchars=\\\{\}]
[[ 2  4  6  8]
 [10 12 14 16]
 [18 20 22 24]
 [26 28 30 32]]
[[0 0 0 0]
 [0 0 0 0]
 [0 0 0 0]
 [0 0 0 0]]
[[  1   4   9  16]
 [ 25  36  49  64]
 [ 81 100 121 144]
 [169 196 225 256]]
[[1. 1. 1. 1.]
 [1. 1. 1. 1.]
 [1. 1. 1. 1.]
 [1. 1. 1. 1.]]
    \end{Verbatim}

    \begin{tcolorbox}[breakable, size=fbox, boxrule=1pt, pad at break*=1mm,colback=cellbackground, colframe=cellborder]
\prompt{In}{incolor}{180}{\boxspacing}
\begin{Verbatim}[commandchars=\\\{\}]
\PY{n+nb}{print}\PY{p}{(}\PY{n}{np}\PY{o}{.}\PY{n}{power}\PY{p}{(}\PY{n}{arr1}\PY{p}{,} \PY{l+m+mi}{2}\PY{p}{)}\PY{p}{)}
\PY{n+nb}{print}\PY{p}{(}\PY{n}{np}\PY{o}{.}\PY{n}{mod}\PY{p}{(}\PY{n}{arr1}\PY{p}{,} \PY{l+m+mi}{2}\PY{p}{)}\PY{p}{)}
\end{Verbatim}
\end{tcolorbox}

    \begin{Verbatim}[commandchars=\\\{\}]
[[  1   4   9  16]
 [ 25  36  49  64]
 [ 81 100 121 144]
 [169 196 225 256]]
[[1 0 1 0]
 [1 0 1 0]
 [1 0 1 0]
 [1 0 1 0]]
    \end{Verbatim}

    \hypertarget{statiscal-function}{%
\section{Statiscal Function}\label{statiscal-function}}

    \begin{tcolorbox}[breakable, size=fbox, boxrule=1pt, pad at break*=1mm,colback=cellbackground, colframe=cellborder]
\prompt{In}{incolor}{183}{\boxspacing}
\begin{Verbatim}[commandchars=\\\{\}]
\PY{c+c1}{\PYZsh{} np.max is just an alias for np.amax. This function only works on a single input array and finds the value of maximum element in that entire array (returning a scalar}
\PY{n+nb}{print}\PY{p}{(}\PY{n}{np}\PY{o}{.}\PY{n}{max}\PY{p}{(}\PY{n}{arr1}\PY{p}{,} \PY{l+m+mi}{0}\PY{p}{)}\PY{p}{)}
\PY{c+c1}{\PYZsh{} The default behaviour of np.maximum is to take two arrays and compute their element\PYZhy{}wise maximum}
\end{Verbatim}
\end{tcolorbox}

    \begin{Verbatim}[commandchars=\\\{\}]
[13 14 15 16]
    \end{Verbatim}

    \begin{tcolorbox}[breakable, size=fbox, boxrule=1pt, pad at break*=1mm,colback=cellbackground, colframe=cellborder]
\prompt{In}{incolor}{187}{\boxspacing}
\begin{Verbatim}[commandchars=\\\{\}]
\PY{n}{np}\PY{o}{.}\PY{n}{ptp}\PY{p}{(}\PY{n}{arr1}\PY{p}{,} \PY{n}{axis}\PY{o}{=}\PY{l+m+mi}{1}\PY{p}{)}
\end{Verbatim}
\end{tcolorbox}

            \begin{tcolorbox}[breakable, size=fbox, boxrule=.5pt, pad at break*=1mm, opacityfill=0]
\prompt{Out}{outcolor}{187}{\boxspacing}
\begin{Verbatim}[commandchars=\\\{\}]
array([3, 3, 3, 3])
\end{Verbatim}
\end{tcolorbox}
        
    \begin{tcolorbox}[breakable, size=fbox, boxrule=1pt, pad at break*=1mm,colback=cellbackground, colframe=cellborder]
\prompt{In}{incolor}{189}{\boxspacing}
\begin{Verbatim}[commandchars=\\\{\}]
\PY{n+nb}{print}\PY{p}{(}\PY{n}{np}\PY{o}{.}\PY{n}{mean}\PY{p}{(}\PY{n}{arr1}\PY{p}{)}\PY{p}{)}
\PY{n+nb}{print}\PY{p}{(}\PY{n}{np}\PY{o}{.}\PY{n}{sum}\PY{p}{(}\PY{n}{arr1}\PY{p}{)} \PY{o}{/} \PY{n}{arr1}\PY{o}{.}\PY{n}{size}\PY{p}{)}
\end{Verbatim}
\end{tcolorbox}

    \begin{Verbatim}[commandchars=\\\{\}]
8.5
8.5
    \end{Verbatim}

    \begin{tcolorbox}[breakable, size=fbox, boxrule=1pt, pad at break*=1mm,colback=cellbackground, colframe=cellborder]
\prompt{In}{incolor}{190}{\boxspacing}
\begin{Verbatim}[commandchars=\\\{\}]
\PY{n}{np}\PY{o}{.}\PY{n}{median}\PY{p}{(}\PY{n}{arr1}\PY{p}{)}
\end{Verbatim}
\end{tcolorbox}

            \begin{tcolorbox}[breakable, size=fbox, boxrule=.5pt, pad at break*=1mm, opacityfill=0]
\prompt{Out}{outcolor}{190}{\boxspacing}
\begin{Verbatim}[commandchars=\\\{\}]
8.5
\end{Verbatim}
\end{tcolorbox}
        
    \hypertarget{sorting-and-searching}{%
\section{Sorting and Searching}\label{sorting-and-searching}}

    \begin{tcolorbox}[breakable, size=fbox, boxrule=1pt, pad at break*=1mm,colback=cellbackground, colframe=cellborder]
\prompt{In}{incolor}{195}{\boxspacing}
\begin{Verbatim}[commandchars=\\\{\}]
\PY{n}{arr3} \PY{o}{=} \PY{n}{random}\PY{o}{.}\PY{n}{randint}\PY{p}{(}\PY{o}{\PYZhy{}}\PY{l+m+mi}{5}\PY{p}{,} \PY{l+m+mi}{5}\PY{p}{,} \PY{l+m+mi}{16}\PY{p}{)}
\PY{n+nb}{print}\PY{p}{(}\PY{n}{arr3}\PY{p}{)}
\end{Verbatim}
\end{tcolorbox}

    \begin{Verbatim}[commandchars=\\\{\}]
[-1  3  4  0 -2 -4  2  0  0  0  2 -3  2 -2 -1 -5]
    \end{Verbatim}

    \begin{tcolorbox}[breakable, size=fbox, boxrule=1pt, pad at break*=1mm,colback=cellbackground, colframe=cellborder]
\prompt{In}{incolor}{196}{\boxspacing}
\begin{Verbatim}[commandchars=\\\{\}]
\PY{n+nb}{print}\PY{p}{(}\PY{n}{np}\PY{o}{.}\PY{n}{sort}\PY{p}{(}\PY{n}{arr3}\PY{p}{)}\PY{p}{)}
\end{Verbatim}
\end{tcolorbox}

    \begin{Verbatim}[commandchars=\\\{\}]
[-5 -4 -3 -2 -2 -1 -1  0  0  0  0  2  2  2  3  4]
    \end{Verbatim}

    \begin{tcolorbox}[breakable, size=fbox, boxrule=1pt, pad at break*=1mm,colback=cellbackground, colframe=cellborder]
\prompt{In}{incolor}{198}{\boxspacing}
\begin{Verbatim}[commandchars=\\\{\}]
\PY{n}{np}\PY{o}{.}\PY{n}{nonzero}\PY{p}{(}\PY{n}{arr3}\PY{p}{)}
\end{Verbatim}
\end{tcolorbox}

            \begin{tcolorbox}[breakable, size=fbox, boxrule=.5pt, pad at break*=1mm, opacityfill=0]
\prompt{Out}{outcolor}{198}{\boxspacing}
\begin{Verbatim}[commandchars=\\\{\}]
(array([ 0,  1,  2,  4,  5,  6, 10, 11, 12, 13, 14, 15], dtype=int64),)
\end{Verbatim}
\end{tcolorbox}
        
    \begin{tcolorbox}[breakable, size=fbox, boxrule=1pt, pad at break*=1mm,colback=cellbackground, colframe=cellborder]
\prompt{In}{incolor}{200}{\boxspacing}
\begin{Verbatim}[commandchars=\\\{\}]
\PY{n}{arr4} \PY{o}{=} \PY{n}{arr3} \PY{o}{\PYZgt{}} \PY{l+m+mi}{1}
\PY{n}{arr4}
\end{Verbatim}
\end{tcolorbox}

            \begin{tcolorbox}[breakable, size=fbox, boxrule=.5pt, pad at break*=1mm, opacityfill=0]
\prompt{Out}{outcolor}{200}{\boxspacing}
\begin{Verbatim}[commandchars=\\\{\}]
array([False,  True,  True, False, False, False,  True, False, False,
       False,  True, False,  True, False, False, False])
\end{Verbatim}
\end{tcolorbox}
        
    \hypertarget{copy-and-view}{%
\section{Copy and view}\label{copy-and-view}}

    \begin{tcolorbox}[breakable, size=fbox, boxrule=1pt, pad at break*=1mm,colback=cellbackground, colframe=cellborder]
\prompt{In}{incolor}{202}{\boxspacing}
\begin{Verbatim}[commandchars=\\\{\}]
\PY{n}{copy\PYZus{}arr} \PY{o}{=} \PY{n}{arr1}\PY{o}{.}\PY{n}{copy}\PY{p}{(}\PY{p}{)}
\PY{n}{view\PYZus{}arr} \PY{o}{=} \PY{n}{arr1}\PY{o}{.}\PY{n}{view}\PY{p}{(}\PY{p}{)}
\PY{n+nb}{print}\PY{p}{(}\PY{l+s+s2}{\PYZdq{}}\PY{l+s+s2}{origin: }\PY{l+s+s2}{\PYZdq{}}\PY{p}{,} \PY{n+nb}{id}\PY{p}{(}\PY{n}{arr1}\PY{p}{)}\PY{p}{)}
\PY{n+nb}{print}\PY{p}{(}\PY{l+s+s2}{\PYZdq{}}\PY{l+s+s2}{copy: }\PY{l+s+s2}{\PYZdq{}}\PY{p}{,} \PY{n+nb}{id}\PY{p}{(}\PY{n}{copy\PYZus{}arr}\PY{p}{)}\PY{p}{)}
\PY{n+nb}{print}\PY{p}{(}\PY{l+s+s2}{\PYZdq{}}\PY{l+s+s2}{view: }\PY{l+s+s2}{\PYZdq{}}\PY{p}{,} \PY{n+nb}{id}\PY{p}{(}\PY{n}{view\PYZus{}arr}\PY{p}{)}\PY{p}{)}
\end{Verbatim}
\end{tcolorbox}

    \begin{Verbatim}[commandchars=\\\{\}]
origin:  2859663134768
copy:  2859663040384
view:  2859663037424
    \end{Verbatim}

    \hypertarget{linear-algebra}{%
\section{Linear Algebra}\label{linear-algebra}}

    \begin{tcolorbox}[breakable, size=fbox, boxrule=1pt, pad at break*=1mm,colback=cellbackground, colframe=cellborder]
\prompt{In}{incolor}{203}{\boxspacing}
\begin{Verbatim}[commandchars=\\\{\}]
\PY{n}{arr1}
\end{Verbatim}
\end{tcolorbox}

            \begin{tcolorbox}[breakable, size=fbox, boxrule=.5pt, pad at break*=1mm, opacityfill=0]
\prompt{Out}{outcolor}{203}{\boxspacing}
\begin{Verbatim}[commandchars=\\\{\}]
array([[ 1,  2,  3,  4],
       [ 5,  6,  7,  8],
       [ 9, 10, 11, 12],
       [13, 14, 15, 16]])
\end{Verbatim}
\end{tcolorbox}
        
    \hypertarget{io-with-numpy}{%
\section{I/O with Numpy}\label{io-with-numpy}}

    \begin{tcolorbox}[breakable, size=fbox, boxrule=1pt, pad at break*=1mm,colback=cellbackground, colframe=cellborder]
\prompt{In}{incolor}{206}{\boxspacing}
\begin{Verbatim}[commandchars=\\\{\}]
\PY{n}{np}\PY{o}{.}\PY{n}{save}\PY{p}{(}\PY{l+s+s2}{\PYZdq{}}\PY{l+s+s2}{out.txt}\PY{l+s+s2}{\PYZdq{}}\PY{p}{,} \PY{n}{arr1}\PY{p}{)}
\end{Verbatim}
\end{tcolorbox}

    \begin{tcolorbox}[breakable, size=fbox, boxrule=1pt, pad at break*=1mm,colback=cellbackground, colframe=cellborder]
\prompt{In}{incolor}{207}{\boxspacing}
\begin{Verbatim}[commandchars=\\\{\}]
\PY{n}{np}\PY{o}{.}\PY{n}{load}\PY{p}{(}\PY{l+s+s2}{\PYZdq{}}\PY{l+s+s2}{out.txt.npy}\PY{l+s+s2}{\PYZdq{}}\PY{p}{)}
\end{Verbatim}
\end{tcolorbox}

            \begin{tcolorbox}[breakable, size=fbox, boxrule=.5pt, pad at break*=1mm, opacityfill=0]
\prompt{Out}{outcolor}{207}{\boxspacing}
\begin{Verbatim}[commandchars=\\\{\}]
array([[ 1,  2,  3,  4],
       [ 5,  6,  7,  8],
       [ 9, 10, 11, 12],
       [13, 14, 15, 16]])
\end{Verbatim}
\end{tcolorbox}
        
    \begin{tcolorbox}[breakable, size=fbox, boxrule=1pt, pad at break*=1mm,colback=cellbackground, colframe=cellborder]
\prompt{In}{incolor}{208}{\boxspacing}
\begin{Verbatim}[commandchars=\\\{\}]
\PY{n}{np}\PY{o}{.}\PY{n}{savetxt}\PY{p}{(}\PY{l+s+s2}{\PYZdq{}}\PY{l+s+s2}{outtxt.txt}\PY{l+s+s2}{\PYZdq{}}\PY{p}{,} \PY{n}{arr1}\PY{p}{)}
\end{Verbatim}
\end{tcolorbox}

    \begin{tcolorbox}[breakable, size=fbox, boxrule=1pt, pad at break*=1mm,colback=cellbackground, colframe=cellborder]
\prompt{In}{incolor}{209}{\boxspacing}
\begin{Verbatim}[commandchars=\\\{\}]
\PY{n}{np}\PY{o}{.}\PY{n}{loadtxt}\PY{p}{(}\PY{l+s+s2}{\PYZdq{}}\PY{l+s+s2}{outtxt.txt}\PY{l+s+s2}{\PYZdq{}}\PY{p}{)}
\end{Verbatim}
\end{tcolorbox}

            \begin{tcolorbox}[breakable, size=fbox, boxrule=.5pt, pad at break*=1mm, opacityfill=0]
\prompt{Out}{outcolor}{209}{\boxspacing}
\begin{Verbatim}[commandchars=\\\{\}]
array([[ 1.,  2.,  3.,  4.],
       [ 5.,  6.,  7.,  8.],
       [ 9., 10., 11., 12.],
       [13., 14., 15., 16.]])
\end{Verbatim}
\end{tcolorbox}
        

    % Add a bibliography block to the postdoc
    
    
    
\end{document}
